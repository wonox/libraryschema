\documentclass[submit,noauthor]{ono}
% \documentclass[submit,techrep,noauthor]{ipsj}
%\documentclass{ipsj}
\usepackage[dvipdfmx]{graphicx}
\usepackage{graphicx}
\usepackage{float}
\usepackage{url}
%% Bookmarkの文字化け対策(日本語向け)
\PassOptionsToPackage{hyphens}{url}

\usepackage{lmodern}
\usepackage{booktabs}
% \usepackage[tablesfirst]{endfloat}
% \usepackage{endfloat}
\usepackage{tabularx}

\def\Underline{\setbox0\hbox\bgroup\let\\\endUnderline}
\def\endUnderline{\vphantom{y}\egroup\smash{\underline{\box0}}\\}
\def\|{\verb|}

\begin{document}

\title{大学図書館の概念スキーマ(ライブラリー・スキーマ)の設計\\}

\etitle{Design of a conceptual schema for a Academic library(Library Schema) \\}

\paffiliate{JU}{人間文化研究機構\\
	National Institutes for the Humanities}

\author{小野 亘}{Wataru, ONO}{https://orcid.org/0000-0002-6398-9317}[ono@nihu.jp]

\begin{abstract}
本稿は、「オープンサイエンス時代における大学図書館の在り方について(審議のまとめ)」で提案されたライブラリー・スキーマを、大学図書館を一つのデータベースとした場合の概念スキーマと考え、そのスキーマを試作し、それによって大学図書館の論理構造について考察を行ったものである。ライブラリー・スキーマを試作することで、大学図書館全体のコンテンツ(印刷体、研究データ、ボーンデジタルなど)の集合的な管理とアクセス、異なる図書館やリポジトリの使用条件やアクセス方法がデジタル的に一貫して管理されることにより、大学図書館全体が統一的なサービスとして機能できることについて考察した。
\end{abstract}

	\begin{jkeyword}
		ライブラリー・スキーマ, 大学図書館, 学術図書館, 概念スキーマ
	\end{jkeyword}
	
\begin{eabstract}
	This paper considers the library schema proposed in "The University Library in the Age of Open Science (Summary of Discussion)" as a conceptual schema for the university library as a single database, and examines the logical design of the university library by prototyping this schema
	By prototyping the library schema, I discussed the collective management of and access to the contents of the whole university library, including printed materials, research data, and bone digital materials, and how the conditions of use and access methods for different libraries and repositories can be managed digitally and consistently, so that the entire university library can function as a unified service.
\end{eabstract}
	
	\begin{ekeyword}
		Library Schema, University Libraries, Academic Libraries, Conceptual Schema
	\end{ekeyword}
	
	\maketitle
	
%1
\section{はじめに}
	
本稿では、「オープンサイエンス時代における大学図書館の在り方について(審議のまとめ)」\cite{まとめ}(以下、「審議のまとめ」という。)で提案されているライブラリー・スキーマを、大学図書館を一つのデータベースとした場合の概念スキーマと捉え、概念スキーマとしてのライブラリー・スキーマを、図\ref{fig:library_schma} のように試作し、それによって大学額図書館の論理構造について考察を行った。

本稿では、大学に附属する図書館としての大学図書館だけでなく、広く学術図書館として大学図書館という用語を使用している。

なお、筆者及び本稿で述べる見解は、審議のまとめをまとめられた委員の先生方や文部科学省の見解とはまったく関係がなく、個人的に示唆を受けたこと以上のものではないことを付言する。

%2
\section{方法}
図\ref{fig:library_schma} は、Markdownテキストでフローチャートなどを描くことができるMermaid記法により作成し、ソースコードは	以下のGithub 
\footnotetext{https://github.com/wonox/libraryschema}
に置いた。
	
%3
\section{結論}
大学図書館を一つのデータベースと見立て、概念スキーマを試作することによって、大学図書館全体として、印刷体、研究データやボーンデジタルを含む、あらゆるコンテンツについて、集合的な管理、アクセスを考えることが可能である。所蔵している図書館、アーカイブしているリポジトリによって異なるその使用条件(ライセンス)、アクセス方法がデジタル的に一貫性をもって管理されることによって、物理的には、どこの図書館に所蔵されていても、あるいはどこのリポジトリにアーカイブされていても、必要な手続きをとればアクセスが可能なことが(場合によっては可能でないこことも含めて)保証されることが、大学図書館全体が一つのサービスとして存在するためには極めて重要と考える。
	
%4
\section{背景}
%4.1
\subsection{ライブラリー・スキーマとは}

2023年1月に文部科学省科学技術・学術審議会 情報委員会オープンサイエンス時代における大学図書館の在り方検討部会(以下、「検討部会」という。)から審議のまとめ\cite{まとめ} が公表された。
その中で、次のように「ライブラリー・スキーマ」という概念が提案された。

\begin{quote}
	「デジタル・ライブラリー」の実現には、大学図書館機能を物理的な「場」に制約されない形で再定義することが求められる。そのためには、「ライブラリー・スキーマ」を明確にした上で、利用者が何を求めているかを整理・再検討し、それを反映してデザインされた最適な環境を構築する必要がある。
\end{quote}

(中略)

\begin{quote}
	(筆者注:物理的な「場」に制約されることなく大学図書館機能を再定義する)その前提として、様々な利用者に適した図書館のサービスをデザインするために必要な、自らの存在を規定する基本的な論理構造としての「ライブラリー・スキーマ」を明確にする必要がある。「ライブラリー・スキーマ」が実際にどのように見えるかは、研究あるいは教育の文脈、分野や立場(教員か学生かなど)によって異なっており、特に今後、仮想的な空間において大学図書館機能の実現を図る際には、その点に十分な留意が必要である。将来的には、利用者の立場ごとに異なる仮想空間(メタバース)を設けて、「ライブラリー・スキーマ」と接続することが想定される。
\end{quote}

筆者は、このライブラリー・スキーマという考え方に示唆を受けて、筆者なりのライブラリー・スキーマを設計することを試みた。
その際、大学図書館を一つのシステム、全体を一つのデータベースと考え、そのデータベーススキーマとして考えることができるのではないか、と仮定した。

	%4.2
	\subsection{スキーマについて}

	ライブラリー・スキーマ は、審議のまとめ\cite{まとめ} の中では明示されていないが、検討部会の議事録\cite{議事録}では以下のとおり「ビュー」という用語と共に語られていることなどを考え合わせると、データベース用語でいう概念スキーマのアナロジーと考えられる。もしくは、そう考えると理解しやすいと考えた。

	なお、


\begin{quote}
図書館について、ユーザービューが1つや2つじゃないですよね、教育・研究の現場は。それは学生のビューであるとともに、教える側のビューであり、また分野のいろんなビューがあって、それらを今までは1つの図書館機能として見せて実現してきたけど、デジタル化に伴って複数の顔を見せられるようになっています。そうはいうものの、図書館を主体としたときの論理構造は1つで、それが複数のビューに対応できるものになっていくのじゃないかというお話です。
\end{quote}
(中略)
\begin{quote}
従来の図書館が建って本が並んでいて、司書さん・事務の方がいるという、そこだけでは済まなくなっていることは確かで、そうした構造を抽象化した論理構造(スキーマ)とともに、さらに上部にユーザービューに応じた論理構造(サブスキーマ)が入るんじゃないかなという気はしています。
\end{quote}
(中略)
\begin{quote}
一人一人の人の様々なニーズに応えるからということでいろんなことを何も考えずにやってしまうと、多分システムはシステム全体としての合理性というものを全く追求できなくなってしまう可能性があって、そういった枠組みをきちんと決めるものがここで言うところのライブラリ・スキーマと言われるものなのではないか
\end{quote}
(中略)
\begin{quote}
ライブラリ・スキーマという言葉、・・・(中略)・・・ モデルを構築してつくっていくものだというふうに認識しています。同時にこれは、図書館の現実とここでの議論をふまえて鍛えていくものでもありましょう。
\end{quote}
(中略)
\begin{quote}
ユーザービュー、あるいはサブスキーマ、そういうものが上にあって、下には人間と建物と本、そして情報処理装置からなる図書館の構造があって、その間に立つものとしてライブラリ・スキーマを設定する。どこまできれいにできるかというのは分からないですけど、まずはモデルとしてそういう3層構造として定義されるのじゃないか
\end{quote}

	%4.3
	\subsection{概念スキーマ}

スキーマ(schema)とは、wikipedia によると以下のように説明されている。

	\begin{quote}
	データベースの構造であり、データベース管理システム (DBMS) でサポートされている形式言語で記述される。関係データベースでは、スキーマは関係 (表) と関係内の属性 (フィールド) 、属性や関係の関連の定義である。
	\end{quote}
	
スキーマの中でも、米国国家規格協会(ANSI)が1975年に提唱した「ANSI/SPARC 3層スキーマ」\cite{ansi} が有名である。ANSIの3層スキーマは、外部レベル、概念レベル、内部レベルの3つのスキーマに分かれており、その内、概念スキーマは、ハードウェアやソフトウェアに依存せずに、データベースにどのようなデータが格納され、そのデータがどのように相互に関連付けられているか、ユーザの構造、	データベースのグローバルビューなどがが記述される。

	%4.4
	\subsection{一つのデータベースとしての大学図書館}

審議のまとめ \cite{まとめ} に、以下のとおりあることからも、大学図書館を連携する一つの大きなデータベースとして考え、そのデータの流れを考えることは有効である。

\begin{quote}
「デジタル・ライブラリー」においては、「一大学で完結する形で一つの図書館システムを整備する」という従来の前提にとらわれる必要はない。例えば、デジタルコンテンツを扱うプラットフォームの共有化、異なるプラットフォームの相互連携、コンテンツ利用契約の統合化、図書館システムの共同運用など「デジタル・ライブラリー」を実現するために大学間で連携して取り組むべき課題は多数存在する。その際、データセントリックな考え方に立って連携を構想することが重要である。
\end{quote}

%5
\section{大学図書館の論理構造}

%5.1
\subsection{データの流れ}

大学図書館を、そのデータの流れから考えると、おおよその流れは以下のようになると考える(数字は必ずしもワークフローの順序ではない)。

\begin{enumerate}
  \item 研究者等が作ったデータ・コンテンツを、アーカイブする
	\item アーカイブしたコンテンツにメタデータを付与
  \item メタデータを検索に渡すとともに、外部にも流通
  \item ユーザは、メタデータを検索
  \item ユーザの属性と、データ・コンテンツの属性にしたがって、データ・コンテンツを利用可能かどうか認証・認可
  \item 認証・認可された内容にしたがってデータ・コンテンツを利用
\end{enumerate}

これはデータの流れを抽象化したものであり、本稿では、以下のように定義する。

\begin{itemize}
  \item アーカイブ: 物理的に所蔵・配架すること、データとして保存されることを含み、場所は自館内もあれば、他館や、外部のストレージ、出版者が保存することなども含む
  \item メタデータ: いわゆる書誌・目録の他、アーカイブから引き出すための所蔵・アクセス情報、ライセンス情報などを含む。物理的に配架されている場合、禁帯出などもライセンス情報の一種
  \item 認証・認可: データ・コンテンツを要求する主体がどこの誰かを認証し、そのデータ・コンテンツはどういう属性の主体にはどういうアクセスが許可されているか、を判断する過程
\end{itemize}

このように定義することによって、リアルな図書館と、デジタルライブラリとを、さらには仮想空間(メタバース)であっても、大学図書館を共通の構造として考えることが可能になる。

%5.2
\subsection{考察}

上記のように、大学図書館の論理構造を捉えると、大学図書館の機能を、以下のように考えることができる。

\begin{description}
  \item[publish] publish という機能を、コンテンツのアクセス可能化とそれに紐づくメタデータ管理、ライセンス管理、アクセス制御と捉えることができる。
  \begin{itemize}
	\item アクセス可視化とは、著作権法にいう公表、送信可能化に加え、実際に公衆にアクセスが可能であることを周知する行為と捉える。
  \item publish と、アーカイブ機能を、論理的には切り離すことができる。
  \item  昔ながらの紙の図書も、電子ジャーナル/ブックもリポジトリや研究データもデジタルアーカイブも共通に考えることができる。
  \item これによって、研究データの publish、機関リポジトリによる研究成果のpublish も、本来的な大学図書館機能の一つである、と考えることができる。
  \item  商業的出版者のみが担ってきた部分と、大学図書館の出版機能も共通に考えることができる。
  \item  単なるアイデアではあるが、例えば、出版において、大学図書館は永続的なアーカイブとオンラインアクセスの提供を担い、それ以外の部分を商業的出版者が担う、というような機能分離も考えられないか。商業的出版者は、そのコストについて大学や著者からのコスト負担(支払い)を受けることも考えられ、あるいは紙媒体の販売など付加価値を付けた販売による収益を得ることなども考えられるのではないか。
\end{itemize}
\end{description}

\begin{description}
  \item[資料提供] 大学図書館での資料提供という機能を、ユーザの認証・認可と、コンテンツ・データのライセンスとステイタス管理に基づくアクセス(物流)制御と、その結果による提供方法の選択、と捉える。
\begin{itemize}
	\item 所蔵に紐づくライセンスデータと、それを認証・認可に紐づけると、貸出とかアクセス管理になる。
	\item 入館ゲートは入館資格の確認という 認証・認可であり、それに基づいて、閲覧という資料への アクセス(物流)制御 が行われている。
	\item 閉架書庫に入庫できるか、などもまったく同様である。
	\item オンラインにおいては、IPアドレス認証や、各種のID/パスワード認証により 認証・認可が行われ、それに基づいて、オープンアクセスや各種の条件付き制限提供などを含む アクセス(物流)制御 が行われている。
\end{itemize}
\end{description}

\begin{description}
  \item[所蔵/所在] 所在データとナレッジベース管理は、そのコンテンツ・データが、どこにかつどういう条件でアーカイブされているか、どこに行けば、どういう条件でアクセスできるか、という意味で共通の情報である。
  \begin{itemize}
		\item 書架のどこにあるか、開架書架か閉架書庫か、貴重書庫か。
		\item 海外のオープンなリポジトリにオープンアクセスとして提供されているのか、自館内にあるリポジトリにある提供条件のもと提供委されているのか、など。
\end{itemize}
\end{description}

\begin{description}
  \item[検索] 大学図書館における検索を、書誌的メタデータから検索インデクスを作製し、その検索インデクスをユーザが検索し、検索結果から、所在データ/ナレッジベースを手繰り寄せて、それによりライセンスにしたがってコンテンツ・データを入手する過程と捉える。
	\begin{itemize}
		\item 書誌的メタデータは、コンテンツ・データの作成者が付与したものに、大学図書館が必要に応じ情報を付加したものである
		\item 紙の図書の場合、表題紙や奥付に作成者が付与したメタデータが記載されており、それをもとに大学図書館等が目録を作っている
    \item  商業出版者による電子ジャーナル、電子ブックの場合、作成者と出版者により検索インデクス自体が提供されている場合が多い
		\item 研究データ等を自機関のリポジトリで提供する場合、研究者が付与したメタデータを基に大学図書館等がメタデータを作成する。
		\item ライセンスデータについても上記と同様である。
		\item 書誌データそのもの、検索インデクスと、所蔵データ/ナレッジベース、ライセンスは、ID等によって連携されてさえいれば、物理的に同じデータベース内で密結合している必要はない。
		\end{itemize}
	\end{description}

	\begin{description}
		\item[集合的コレクション] 上記が揃うことによって、シェアードプリントや、Collective Collection(集合的コレクション)、あるいは統合的発見・アクセス環境の考え方の基礎となる。
		\begin{itemize} 
			\item  どこの図書館あるいは図書館でなくともどこにどういう条件でアクセスできるかが、連携されていれば、分散的なシェアードプリントや、Collective Collectionとして機能する。
			\item シェアードプリントや、Collective Collectionは、もともと印刷体をデジタル化することによって集団的に管理することを中心に提唱、実践されてきたが、 \cite{dempsey} 大学図書館全体を一つのデータベースとして考えることは、研究データやボーンデジタルを含む、あらゆるコンテンツについて集団的な管理、アクセスを考えることになる。所蔵している図書館、アーカイブしているリポジトリによって、その使用条件(ライセンス)が異なるのは当然だたが、その使用条件とアクセス方法がデジタル的に一貫性をもって管理されることによって、物理的には、どこの図書館に所蔵されていても、あるいはどこのリポジトリにアーカイブされていても、必要な手続きをとればアクセスが可能なことが(場合によっては可能でないこことも含めて)保証されることが、大学図書館全体が一つのサービスとして存在するためには極めて重要と考える。
\end{itemize}
\end{description}

%5.3
\subsection{その他の図書館的機能}

審議のまとめ \cite{まとめ} では、その他の図書館的機能について、以下のとおり述べられている。

\begin{quote}
学修環境整備に関する既存業務のうち、主に大学図書館が担ってきた部分については、これまでの活動の評価を踏まえ、大学図書館が引き続き行うかどうか改めて整理する等、大学全体で検討する。
\end{quote}

ライブラリー・スキーマは、これまで検討してきたように、大学図書館を単館のみならず(単館に適用することも可能だが)大学図書館全体を一つのデータベースとして捉えた場合の、概念スキーマである。

しかし、そのデータベースをどう使ってもらうか、使い方をどう案内するか、使ってどういう成果を出すか、というのはスキーマの外部の課題である。当然、それらを考えあわせた上で、データベースは設計され、通常データベースを作るとすれば、解決すべき課題があって、そのためにデータベースは作られる。

従って、大学、あるいは学術コミュニティにとっての研究・教育を支援するために、大学図書館(というデータベース)が必要であり、そのための(再)設計をすることがライブラリー・スキーマの設計であろう。

審議のまとめ\cite{まとめ} に以下のとおり挙げられているように、研究者や学生のコンテンツ・データ作成支援、場所の提供、検索やICTのインストラクションを含むいわゆる情報リテラシーの提供なども、必ずしも大学図書館の内部に抱え込むことは、少なくとも必然的ではない。ライブラリ・スキーマでそれぞれの機能を抽象化したように、大学の研究・教育を支援するための機能として抽象化し、再設計する必要があるだろう。

\begin{quotation}
教材作成における著作物の利用を支援する体制の構築などに新たに取り組むことになる。その際、例えばラーニング・コモンズなど、これまで大学図書館が主導的な役割を果たしてきた学修環境整備にかかる活動についても、その成果を評価した上で見直しを行うことなどが求められる。大学図書館はデジタル化されたコンテンツの利活用をその機能の中核に据える一方、学修環境については大学全体として再構築することが望まれる。
\end{quotation}

% %6
\section{おわりに}
ライブラリー・スキーマは、大学図書館全体を一つのデータベースとして捉え、データベースの考え方の一つである3層スキーマの一つである概念スキーマとして検討した。

審議のまとめ \cite{まとめ} では、``「デジタル・ライブラリー」の実現の際に直面する各課題の解決に向、「一大学一図書館」という前提にとらわれず、例えば、複数の大学図書館で「コンソーシア	ム」を形成するなど、相互運用の観点から連携して対応する。''
と述べられているが、1980年の学術審議会「今後における学術情報システムの在り方について(答申)」のさらに以前より、大学図書館がその必要とする資料やサービスを単館のみで充足できたことはなく、むしろ、連携することによってしか、その機能を果しえなかった。そう考えると、デジタル化によって一つの大学図書館として機能するための摩擦係数が大幅に下がっただけであって、今回検討したようなライブラリー・スキーマも、実は目新しいものではないのかもしれない。

また、今回試作した図\ref{fig:library_schma} に含まれる要素以外にも、図書館的機能を考える上で、不足している要素もあるかもしれないし、要素間の関係性などはさらに検討していく余地は大きいと考えている。 

	% 本稿では,
	
	% \begin{acknowledgment}
	% 謝辞はここにかく
	% \end{acknowledgment}


	%============================
	%--------------
	\onecolumn

	% 2段ぶちぬきで下方に図を表示する
		\begin{figure}[tb]

			% \figref{fig:box1}
			\twocolfig{
				\includegraphics[width=12cm]{./picture/library_schema_mock.png}
			}
			\twocolcaption{大学図書館の概念スキーマ(ライブラリー・スキーマ)の設計	\label{fig:library_schma}}
		\end{figure}
		% <大き目な図これ[b]だとボトムに図が表示される>
		
		
		\twocolumn

	
	\begin{thebibliography}{}
	% 	\bibitem{okumura}
	% 	奥村晴彦:改訂第5版 \LaTeXe 美文書作成入門,
	% 	技術評論社(2010).
		
	% 	\bibitem{companion}
	% 	Goossens, M., Mittelbach, F. and Samarin, A.: {\it The LaTeX Companion},
	% 	Addison Wesley, Reading, Massachusetts (1993).
	
	\bibitem{まとめ}
	科学技術・学術審議会 情報委員会オープンサイエンス時代における大学図書館の在り方検討部会「オープンサイエンス時代における大学図書館の在り方について(審議のまとめ)令和5年1月25日 \url{https://www.mext.go.jp/content/20230325-mxt_jyohoka01-000028544.pdf.pdf}(文部科学省, 2023)

	\bibitem{議事録}
	"オープンサイエンス時代における大学図書館の在り方検討部会(第7回)議事録” (文部科学省, 2022)
	\url{https://www.mext.go.jp/b_menu/shingi/gijyutu/gijyutu29/004/gijiroku/mext_00007.html}
	%\refdatej{ 9 Aug. 2023.}
	
	
	\bibitem{wikipedia}
	ウィキペディア "スキーマ (データベース)" 
	\url{https://ja.wikipedia.org/wiki/%E3%82%B9%E3%82%AD%E3%83%BC%E3%83%9E_\(%E3%83%87%E3%83%BC%E3%82%BF%E3%83%99%E3%83%BC%E3%82%B9\)}
	\refdatej{ 9 Aug. 2023.}

	\bibitem{ansi}
	ANSI/X3/SPARC Study Group on Data Base Management Systems: (1975), Interim Report. FDT, ACM SIGMOD bulletin. Volume 7, No. 2 
	url{https://dl.acm.org/toc/sigmod/1975/7/2}
	\refdatej{ 9 Aug. 2023.}

	\bibitem{dempsey}
	Dempsey, Lorcan, Brian Lavoie, Constance Malpas, Lynn Silipigni Connaway, Roger C.
Schonfeld, JD Shipengrover, and Günter Waibel. 2013. Understanding the Collective
Collection: Towards a System-wide Perspective on Library Print Collections. Dublin,
Ohio: OCLC Research.
	url{https://doi.org/10.25333/E94Q-9Q39}
	\refdatej{ 9 Aug. 2023.}

	
	% 	\bibitem{book2}
	% 	Strunk, W.J. and White, E.B.: {\it The Elements of Style, Forth Edition},
	% 	Longman (2000).
		
	% 	\bibitem{book3}
	% 	Blake, G. and Bly, R.W.: {\it The Elements of Technical Writing},
	% 	Longman (1993).
		
	% 	\bibitem{book4}
	% 	Higham, N.J.:
	% 	{\it Handbook of Writing for the Mathematical Sciences},
	% SIAM (1998).
		
	% 	\bibitem{webpage1}
	% 	情報処理学会論文誌ジャーナル編集委員会:
	% 	投稿者マニュアル(オンライン),
	% 	\urlj{httap://www.ipsj.or.jp/journal/ submit/manual/j\_manual.html}%
	% 	\refdatej{2007-04-05}.
		
	% 	\bibitem{webpage2}
	% 	情報処理学会論文誌ジャーナル編集委員会:
	% 	べからず集(オンライン),
	% 	\urlj{http://www.ipsj.or.jp/journal/\\ manual/bekarazu.html}%
	% 	\refdatej{2011-09-15}.
		
	\end{thebibliography}
		

	
	
	%====================
	
	\end{document}
	